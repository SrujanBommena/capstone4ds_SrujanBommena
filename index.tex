% Options for packages loaded elsewhere
\PassOptionsToPackage{unicode}{hyperref}
\PassOptionsToPackage{hyphens}{url}
\PassOptionsToPackage{dvipsnames,svgnames,x11names}{xcolor}
%
\documentclass[
  letterpaper,
  DIV=11,
  numbers=noendperiod]{scrartcl}

\usepackage{amsmath,amssymb}
\usepackage{iftex}
\ifPDFTeX
  \usepackage[T1]{fontenc}
  \usepackage[utf8]{inputenc}
  \usepackage{textcomp} % provide euro and other symbols
\else % if luatex or xetex
  \usepackage{unicode-math}
  \defaultfontfeatures{Scale=MatchLowercase}
  \defaultfontfeatures[\rmfamily]{Ligatures=TeX,Scale=1}
\fi
\usepackage{lmodern}
\ifPDFTeX\else  
    % xetex/luatex font selection
\fi
% Use upquote if available, for straight quotes in verbatim environments
\IfFileExists{upquote.sty}{\usepackage{upquote}}{}
\IfFileExists{microtype.sty}{% use microtype if available
  \usepackage[]{microtype}
  \UseMicrotypeSet[protrusion]{basicmath} % disable protrusion for tt fonts
}{}
\makeatletter
\@ifundefined{KOMAClassName}{% if non-KOMA class
  \IfFileExists{parskip.sty}{%
    \usepackage{parskip}
  }{% else
    \setlength{\parindent}{0pt}
    \setlength{\parskip}{6pt plus 2pt minus 1pt}}
}{% if KOMA class
  \KOMAoptions{parskip=half}}
\makeatother
\usepackage{xcolor}
\setlength{\emergencystretch}{3em} % prevent overfull lines
\setcounter{secnumdepth}{-\maxdimen} % remove section numbering
% Make \paragraph and \subparagraph free-standing
\makeatletter
\ifx\paragraph\undefined\else
  \let\oldparagraph\paragraph
  \renewcommand{\paragraph}{
    \@ifstar
      \xxxParagraphStar
      \xxxParagraphNoStar
  }
  \newcommand{\xxxParagraphStar}[1]{\oldparagraph*{#1}\mbox{}}
  \newcommand{\xxxParagraphNoStar}[1]{\oldparagraph{#1}\mbox{}}
\fi
\ifx\subparagraph\undefined\else
  \let\oldsubparagraph\subparagraph
  \renewcommand{\subparagraph}{
    \@ifstar
      \xxxSubParagraphStar
      \xxxSubParagraphNoStar
  }
  \newcommand{\xxxSubParagraphStar}[1]{\oldsubparagraph*{#1}\mbox{}}
  \newcommand{\xxxSubParagraphNoStar}[1]{\oldsubparagraph{#1}\mbox{}}
\fi
\makeatother


\providecommand{\tightlist}{%
  \setlength{\itemsep}{0pt}\setlength{\parskip}{0pt}}\usepackage{longtable,booktabs,array}
\usepackage{calc} % for calculating minipage widths
% Correct order of tables after \paragraph or \subparagraph
\usepackage{etoolbox}
\makeatletter
\patchcmd\longtable{\par}{\if@noskipsec\mbox{}\fi\par}{}{}
\makeatother
% Allow footnotes in longtable head/foot
\IfFileExists{footnotehyper.sty}{\usepackage{footnotehyper}}{\usepackage{footnote}}
\makesavenoteenv{longtable}
\usepackage{graphicx}
\makeatletter
\def\maxwidth{\ifdim\Gin@nat@width>\linewidth\linewidth\else\Gin@nat@width\fi}
\def\maxheight{\ifdim\Gin@nat@height>\textheight\textheight\else\Gin@nat@height\fi}
\makeatother
% Scale images if necessary, so that they will not overflow the page
% margins by default, and it is still possible to overwrite the defaults
% using explicit options in \includegraphics[width, height, ...]{}
\setkeys{Gin}{width=\maxwidth,height=\maxheight,keepaspectratio}
% Set default figure placement to htbp
\makeatletter
\def\fps@figure{htbp}
\makeatother

\KOMAoption{captions}{tableheading}
\makeatletter
\@ifpackageloaded{caption}{}{\usepackage{caption}}
\AtBeginDocument{%
\ifdefined\contentsname
  \renewcommand*\contentsname{Table of contents}
\else
  \newcommand\contentsname{Table of contents}
\fi
\ifdefined\listfigurename
  \renewcommand*\listfigurename{List of Figures}
\else
  \newcommand\listfigurename{List of Figures}
\fi
\ifdefined\listtablename
  \renewcommand*\listtablename{List of Tables}
\else
  \newcommand\listtablename{List of Tables}
\fi
\ifdefined\figurename
  \renewcommand*\figurename{Figure}
\else
  \newcommand\figurename{Figure}
\fi
\ifdefined\tablename
  \renewcommand*\tablename{Table}
\else
  \newcommand\tablename{Table}
\fi
}
\@ifpackageloaded{float}{}{\usepackage{float}}
\floatstyle{ruled}
\@ifundefined{c@chapter}{\newfloat{codelisting}{h}{lop}}{\newfloat{codelisting}{h}{lop}[chapter]}
\floatname{codelisting}{Listing}
\newcommand*\listoflistings{\listof{codelisting}{List of Listings}}
\makeatother
\makeatletter
\makeatother
\makeatletter
\@ifpackageloaded{caption}{}{\usepackage{caption}}
\@ifpackageloaded{subcaption}{}{\usepackage{subcaption}}
\makeatother

\ifLuaTeX
  \usepackage{selnolig}  % disable illegal ligatures
\fi
\usepackage{bookmark}

\IfFileExists{xurl.sty}{\usepackage{xurl}}{} % add URL line breaks if available
\urlstyle{same} % disable monospaced font for URLs
\hypersetup{
  pdftitle={capstone propoosal},
  colorlinks=true,
  linkcolor={blue},
  filecolor={Maroon},
  citecolor={Blue},
  urlcolor={Blue},
  pdfcreator={LaTeX via pandoc}}


\title{capstone propoosal}
\author{}
\date{}

\begin{document}
\maketitle


\section{CAPSTONE PROPOSAL}\label{capstone-proposal}

\paragraph{Names of your teammates, if any, or specify if you are
working
independently.}\label{names-of-your-teammates-if-any-or-specify-if-you-are-working-independently.}

\begin{itemize}
\tightlist
\item
  Srujan Bommena (myself)
\item
  Yashwanth krishna
\item
  Sathwika Bairi
\end{itemize}

\paragraph{Research topic you will focus on for the remainder of the
semester}\label{research-topic-you-will-focus-on-for-the-remainder-of-the-semester}

Machine Learning and Predictive Analytics: Developing models to predict
outcomes based on historical data.

\paragraph{Mathematical/Statistical methods you plan to use in the
capstone}\label{mathematicalstatistical-methods-you-plan-to-use-in-the-capstone}

\subparagraph{REGRESSION MODELS:}\label{regression-models}

Linear Regression: Model relationships between dependent and independent
variables.

Multiple Regression: Extends linear regression with multiple independent
variables.

Logistic Regression: Used for binary classification problems (e.g.,
fraud detection).

\subparagraph{UNSUPERVISED :}\label{unsupervised}

k-Means Clustering: Groups data into k clusters based on distance
metrics.

Principal Component Analysis (PCA): Dimensionality reduction technique
using eigenvalues and eigen vectors.

Auto-encoders: Deep learning models for feature extraction.

PROBABILISTIC MODELS Bayesian Inference: Updates probabilities as new
evidence is\\
introduced.

Hidden Markov Models (HMM): Used in sequence prediction (e.g., speech
recognition).

Gaussian Mixture Models (GMM): Clustering approach based on probability
distributions

\paragraph{Preferred programming language(s) for the
project}\label{preferred-programming-languages-for-the-project}

R -- Core programming language for data processing, analysis, and
modeling.

tidyverse -- Data wrangling and visualization.

caret -- Model training and evaluation.

gplot2 -- Data visualization.

randomForest, xgboost -- Machine learning algorithms.

shiny -- Interactive dashboards (if required).

SQL -- For querying and managing data sets.

Python (if needed) -- For additional ML modeling and integration

\paragraph{Source and method of obtaining your
dataset}\label{source-and-method-of-obtaining-your-dataset}

Google scholar, Kaggle, twitter

\paragraph{Potential references
(tentative)}\label{potential-references-tentative}

Antons, D., \& Breidbach, C. F. (2018). Big data, big insights?
Advancing service innovation and design with machine learning. Journal
of Service Research, 21, 17--39.
(https://doi.org/10.1177/1094670517738373)

Mohana Chelvan, P., Perumal, K.: A Survey of Feature Selection Stability
Mea-sures. International Journal of Computer and Information Technology
5, Article14 (2016),
https://www.ijcit.com/archives/volume5/issue1/Paper050114.pdf.




\end{document}
